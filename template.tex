\documentclass{beamer}

\usetheme{Pitt}

\title{Pitt Beamer Template}
\subtitle{This is an optional subtitle}
%\date{}
\author[Zac Yu]{Zac Yu (\texttt{zac.yu@pitt.edu})}
\institute{Carnegie Mellon University}

\begin{document}
\maketitle

\section{Styles}

\begin{frame}
	\frametitle{Slide title}
	\framesubtitle{Slide subtitle (optional)}

	This is a sample page that showcases some styles of the theme.\break
	
	\begin{itemize}
		\item Some bullets
		\begin{itemize}
			\item Some sub bullet
			\item Another sub bullet
		\end{itemize}
		\item And another buller
	\end{itemize}

	\begin{enumerate}
		\item Some numbering
		\begin{enumerate}
			\item Some sub numbering
			\item Another sub numbering
		\end{enumerate}
		\item And another numbering
	\end{enumerate}
\end{frame}

\subsection{Blocks}

\begin{frame}
	\frametitle{We can use blocks}

	\begin{block}{A general block}
		Blocks can have any type of text including math $e^{i\pi}-1=0$.
	\end{block}

	\begin{theorem}
		And this is a theorem.
	\end{theorem}

	\begin{proof}
		And this is a proof. Note that the previously theorem is now perfectly proven to be true in all cases.
	\end{proof}

	\begin{definition}
		This is a definition of a new concept.
	\end{definition}
\end{frame}

\subsection{Columns}

\begin{frame}
	\frametitle{Columns should work perfectly too}

	\begin{columns}[onlytextwidth,t]
		\column{0.5\textwidth}
			First column text here

			\only<1>{And overlays should work well also.}
			\only<2>{With text appearing and disappearing!}
		\column{0.5\textwidth}
			Second column here\footnote{Footnotes are also supported!}.
	\end{columns}
\end{frame}

\section{About}

\begin{frame}
	\frametitle{About This Template}
	
	This unofficial Pitt beamer template is based on the (unofficial) CMU beamer theme created by Paulo Casanova and released freely on his website (\url{http://www.cs.cmu.edu/~pcasanov/}).\break
	
	The title page design is inspired by that of lecture slides released by Dr. Nicholas Farnan (\texttt{nlf4@pitt.edu}).\break
	
	{\large \textbf{Disclaimer}\break}
	
	The ``University of Pittsburgh'' name and its logo are trademarks of the University. When using this template, it is \textit{your} responsibility to ensure everything is legal.
\end{frame}

\end{document}
